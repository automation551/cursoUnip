\documentclass[a4paper]{report}
\usepackage[brazil]{babel}
\usepackage[utf8]{inputenc}

\begin{document}

    \title{Portf\'{o}lio\\Atividades Complementares}
    \author{Jo\~{a}o Paulo Dubas\\RA: A54467-4}
    \date{Santos - Praia\\2010}

    \maketitle

    \tableofcontents

    \chapter[Webdesign versus Javascript]{Palestra Impacta: Webdesign versus
    Javascript}

        \section{Introdu\c{c}\~{a}o}
        O objetivo desta palestra foi apresentar as distin\c{c}\~{o}es entre as
        atribui\c{c}\~{o}es do \emph{webdesigner} e do \emph{programador} numa
        equipe de desenvolvimento web, bem como, no mercado atual estas
        duas fun\c{c}\~{o}es acabam se unindo de maneira inadequada.

        Al\'{e}m disso, fez-se a introdu\c{c}\~{a}o de t\'{e}cnicas modernas de
        desenvolvimento javascript, e como est\'{a} linguagem pode ajudar
        o desenvolvedor a criar aplicativos web (\emph{web applications})
        que funcionem de forma similar em todos os navegadores com o
        uso de \emph{frameworks}, tais como, jQuery, Mootools, Pryx
        entre outros.

        \section[Descri\c{c}\~{a}o atividade]{Descri\c{c}\~{a}o da atividade
        \\em ordem cronol\'{o}gica}
        Foi apresentado um hist\'{o}rico bastante rigoroso a respeito da
        evolu\c{c}\~{a}o do processo de desenvovlvimento na internet, desde o
        desenvolvimento da linguagem \emph{javascript}, passando pela
        especifca\c{c}\~{a}o do \emph{CSS}, at\'{e} os dias atuais, onde a
        separa\c{c}\~{a}o entre os \emph{web applications} e os aplicativos
        desktop \'{e} cada vez mais t\^{e}nue.

        Definiu-se os pap\'{e}is ideais do \emph{webdesiger} e do
        \emph{webdeveloper}, conforme apresentado abaixo:

        \begin{description}

            \item[\emph{Webdesigner}] Deve ter como principal meta a de planejar
            a interface com o usu\'{a}rio, a experi\^{e}ncia usu\'{a}rio x
            interface, como as informa\c{c}\~{o}es devem ser dispostas na
            p\'{a}gina e, em doses homeop\'{a}ticas, pode alterar o c\'{o}digo
            fonte do aplicativo.

            \item[\emph{Webdeveloper}] Tem como objetivo criar os mecanismos para
            que as intera\c{c}\~{o}es previstas pelo \emph{webdesigner} tornem-se
            realidade. Para tal utiliza de ferramentas, tais como o javascript,
            para realizar suas tarefas.

        \end{description}

        \'{E} importante notar, que n\~{a}o foram discutidos os pap\'{e}is do
        administrador de sistema, bem como, o desenvolvedor do \emph{backend}.

        Apesar do quadro ideal ter sido apresentado anteriormente, o pan\^{o}rama
        atual mostra, que ambos os profissionais, acabam se mesclando e
        realizando tarefas para as quais n\~{a}o se prepararam, como por exemplo,
        o \emph{webdesigner} tamb\'{e}m trabalha no c\'{o}digo fonte, equanto o
        \emph{webdeveloper} se v\^{e} obrigado a criar a iterface com o
        usu\'{a}rio.

        Feita a rela\c{c}\~{a}o entre o \textbf{ideal} e o \textbf{atual},
        discutiu-se uma proposta de \emph{workflow} para otimizar o trabalho do
        profissional que trabalha com internet. Este fluxo \'{e} apresentado
        abaixo:

        \begin{enumerate}

            \item Defina o foco do trabalho a ser produzido.

            \item Escolha as ferramentas para realizar o trabalho.
                \\Uma bom ambiente integrado de desenvolvimento (\emph{IDE})
                faz toda a diferen\c{c}a.

            \item Crie os \emph{wireframes} e \emph{prot\'{o}tipos} para o
            projeto em trabalho.

            \item Converta os prot\'{o}tipos em algo que o cliente possa
            interagir.

            \item Escolha os frameworks para auxili\'{a}-lo na tarefa de
            desenvolver o trabalho final.

        \end{enumerate}

        Com o \emph{workflow} discutido, apresentou-se uma das \emph{IDE}s da
        Adobe, o \emph{Dreamweaver} em sua vers\~{a} CS5. Um programa excepcionaç
        para o desenvolvedor web\footnote{Apesar de n\~{a}o utiliz\'{a}-lo no meu
        dia a dia, prefiro as solu\c{c}\~{o}es \emph{open-source} como o Vim.}.

        Enfim, discutiram-se alguns dos \emph{frameworks} javascript mais
        utilizados na atualidade, e foram apresentados alguns usos para destes.

        \section{Conclus\~{a}o}

        A palestra foi bastante interessante, n\~{a}o t\~{a}o aprofundada quanto
        eu esperava, mas serviu para mostrar como o mercado de desenvovlvimento
        web, em nosso pa\'{i}s ainda \'{e} imatura frente a outros, al\'{e}m
        disso, foram apresentadas algumas boas pr\'{a}ticas de programa\c{c}\~{a}o.

    \chapter[15\textordmasculine EDTED]{15\textordmasculine Encontro de Design e
    Tecnologia Digital}

        \section{Introdu\c{c}\~{a}o}
        O encontro teve como cen\'{a}rio apresentar tecnologias emergentes,
        tend\^{e}ncias de mercado e prover informa\c{c}oes pr\'{a}ticas para
        seus participantes.

        Os conte\'{u}dos foram dividos em palestras, ministradas por profissionais
        renomados, dentro de sua \'{a}rea de atua\c{c}\~{a}o. Sendo assim, n\~{a}o
        foi poss\'{i}vel atender a todos os temas.

        \section[Descri\c{c}\~{a}o atividade]{Descri\c{c}\~{a}o da atividade
        \\em ordem cronol\'{o}gica}
        
            \subsection[\emph{Product Backlog}]{\emph{Product Backlog}: elabora\c{c}\~{a}o e
            \\manuten\c{c}\~{a}o com garantia de ROI}
            A discuss\~{a}o da garantia de retorno do investimento (ROI) por
            meio da defini\c{c}\~{a}o das atividades a serem realizadas de
            acordo com suas prioridades e como o \emph{Product Owner} e o
            \emph{Scrum Master} devem trabalhar para garantir que as atividades
            definidas no backlog, foi o t\'{o}pico central desta palestra.

            \subsection{Ruby on Rails}

            \subsection{Games}

            \subsection{Acessibilidade na web}

            \subsection{\emph{Start-ups}: Andr\'{e} Felipe - Design Atento}

            \subsection{\emph{Mobile tagging} \& Realidades mistas}

        \section{Conclus\~{a}o}

    \chapter[Python b\'{a}sico]{Python b\'{a}sico para Django e Google App Engine}

        \section{Introdu\c{c}\~{a}o}

        \section[Descri\c{c}\~{a}o atividade]{Descri\c{c}\~{a}o da atividade
        \\em ordem cronol\'{o}gica}

        \section{Conclus\~{a}o}

    \chapter[Django ORM]{Django ORM: SQL sem sujar as m\~{a}os}

        \section{Introdu\c{c}\~{a}o}

        \section[Descri\c{c}\~{a}o atividade]{Descri\c{c}\~{a}o da atividade
        \\em ordem cronol\'{o}gica}

        \section{Conclus\~{a}o}
\end{document}

\chapter{Modelos de Sucesso}
\label{ch:modelo_sucesso}
Conforme apresentado no capítulo anterior, a relação entre informática e
educação vêm se estreitando com o passar do tempo, a seguir vamos apresentar
alguns casos de emprego da informática na educação.

    \section{Portal do professor
            \\(\emph{http://portaldoprofessor.mec.gov.br})}
    Criado em 2008 pelo Ministério de Ciência e Tecnologia, este site tem
    por objetivo apoiar o processo de formação de nossos professores, além
    de aprimorar a prática pedagógica, por meio do compartilhamento de
    conteúdos e aulas, bem como de notícias de interesse aos professores.

    O site é dividido em seis seções:

    \begin{itemize}

        \item Espaço da Aula

        \item Jornal do Professor

        \item Recursos Educacionais

        \item Cursos e Materiais

        \item Interação e Colaboração

        \item Links

    \end{itemize}

        \subsection{Espaço de Aula}
        Nesta área o professor pode compartilhar suas idéias, propostas e
        sugestões metodológicas para o desenvolvimento dos temas
        curriculares propostos, bem como para uso de recursos multimídia e
        ferramentas digitais.

        Neste espaço espera-se criar um intercâmbio de experiências para o
        desenvolvimento criativo de novas estratégias de ensino-aprendizagem.

        \subsection{Jornal do Professor}
        Área do site que possui foco no professor, tendo como proposta
        apresentar o cotidiano da sala de aula, trazendo temas ligados à
        educação.

        \subsection{Recursos Educacionais}
        O conteúdo gerado pelos professores pode ser disponibilizado em
        diferentes mídias, tais como vídeos, animações, simulações, áudios,
        hipertextos, imagens e experimentos práticos. Este conteúdo é
        previamente selecionado para atender todos os componentes
        curriculares e temas relacionados.

        \subsection{Cursos e materiais}
        Nesta seção estão disponíveis informações sobre programas de
        capacitação do Ministério da Educação e Cultura (MEC) e outras
        instituições que oferecem tais programas. Também estão disponíveis
        materiais de estudo como apostilas, estratégias pedagógicas entre
        outros.

        \subsection{Interação e Colaboração}
        Este é um espaço onde o professor pode trocar informações de
        diferentes maneiras e compartilhar seu trabalho. Além de estabelecer
        novos canais de comunicação entre docentes, valorizando suas
        experiências profissionais. Por fim, são publicadas diversas
        ferramentas de interação e colaboração.

        \subsection{Links}
        Nesta área são disponibilizados endereços separados por temática,
        que auxiliam as pesquisas dos professores.

    \section{Banco Internacional de Objetos Educacionais
            \\(\emph{http://objetoseducacionais2.mec.gov.br/})}
    Este site foi desenvolvido em 2008 pelo Ministério da Educação em
    parceria com o Ministério de Ciência e Tecnologia, a Rede
    Latino-americana de Portais Educacionais (RELPE), a Organização dos
    Estados Ibero-americanos (OEI) entre outras.

    O propósito deste site é o de manter e compartilhar recursos
    educacionais digitais de livre acesso e, em diferentes formatos, tais
    como áudio, vídeo, animação, simulação, software educacional, imagens,
    mapas e hipertexto.

    Além disso, este repositório visa fomentar e apoiar experiências de
    diferentes países, com o intuito de nivelar conhecimentos e permitir
    evolução mais rápida no uso de tecnologias em sala de aula.

    Atualmente novos materiais devem ser enviados por correio em CD-ROM ou
    DVD-ROM, onde serão avaliados e catalogados por uma comissão, formada
    por docentes de diferentes universidades.

    \section{Rede Interativa Virtual de Educação
            \\(\emph{http://rived.mec.gov.br/})}
    A Rede Interativa Virtual de Educação (RIVE) é um programa da Secretaria
    de Educação a Distância (SEED), que tem por objetivo o desenvolvimento
    de conteúdos digitais, na forma de objetos de aprendizagem, tais objetos
    visam estimular o raciocínio crítico dos estudantes, associando o
    potencial da informática às novas abordagens pedagógicas.

    Além de promover a produção e disponibilizar na internet os conteúdos
    digitais, a RIVED realiza a capacitação para produção e utilização dos
    objetos de aprendizagem para instituições de ensino superior e rede
    pública de ensino.

        \subsection{Objetos de aprendizagem}
        A RIVED define um objeto de aprendizagem como qualquer recurso que
        possa ser reutilizado para dar suporte ao aprendizado,
        `\emph{quebrando}' o conteúdo educacional em pequenos trechos.

        Os objetos produzidos pela RIVED consistem em atividades multimídia
        e interativas na forma de animações e simulações, que permitem
        testar diferentes caminhos, acompanhar a evolução temporal das
        relações de causa-efeito, visualizar conceitos em diferentes prismas
        e comprovar hipóteses.

    \section{Rede Latino-americana de Portais Educacionais
            \\(\emph{http://www.relpe.org/})}
    Esta rede foi formada em agosto de 2004, por um acordo firmado entre os
    Ministros da Educação de 16 países latino-americanos reunidos em
    Santiago, Chile.

    Sendo composta por portais educativos autônomos, nacionais, de serviço
    público e gratuito, designado para tal efeito pelo Ministro da Educação
    de cada país.

    Cada país é livre para desenvolver seu portal de acordo com seu projeto
    educacional e interesses, utilizando as tecnologias que lhe atenderem
    melhor seus projetos. Sendo que todo o conteúdo criado é disponível
    entre todos os portais membros e são de livre circulação na rede.

    A livre circulação é garantida por uma ferramenta criada pela Fundación
    Chile, chamada de conector, que utiliza o padrão XML para disponibilizar
    a todos os nós da rede os conteúdos produzidos por cada nó. Isso permite
    que o conteúdo seja consumido da maneira que for mais conveniente para
    cada nó.

        \subsection{Objetivo}
        A RELPE foi concebida como um suporte em favor da qualidade e
        igualdade na educação e redução da brecha digital caracterizada nos
        países latino-americanos.

        Os conteúdos gerados são localizados de acordo com as necessidades
        educacionais de cada país participante. Contudo, para garantir a
        indexação, cada nó deve indexar os conteúdos gerados de acordo com
        os padrões adotados pela RELPE e que estão em conformidade com as
        normas internacionais.

        Como suporte à RELPE, cada nó deve produzir uma quantidade anual de
        conteúdos universais estipulada nos acordos de funcionamento anual,
        respeitando os padrões de qualidade definidos. Gerando assim um
        limiar de qualidade que é tido em outros conteúdos oferecidos fora
        da RELPE.

        Sendo assim a RELPE propõe-se a:

        \begin{itemize}

            \item Oferecer aos usuários de cada país um maior número de
            conteúdos adaptados ao seu projeto educacional.

            \item Favorecer o intercambio de conhecimentos e experiências
            sobre o uso educacional das tecnologias de informação e
            comunicação.

            \item Reduzir os custos de produção dos portais nacionais,
            facilitando a criação de tecnologias compartilhadas.

            \item Evoluir de forma conjunta e a partir de financiamento
            multilateral que fortalece os projetos nacionais.

        \end{itemize}

    \section{Robótica nas Escolas}
    Luara Baggi e João Pedro Franch, ambos 13 anos, montaram o robô
    Godofredo e choraram juntos sua morte no ano passado. Planejar e operar
    um engenho digital são apenas duas das atividades da disciplina
    Robótica, que uma escola da rede privada do Distrito Federal incorporou
    a seu currículo a partir de 2006. No Colégio Anima, que fica no
    Plano Piloto, área de classe média alta de Brasília, esse tema faz parte do
    projeto pedagógico. Todo o programa de educação tecnológica é levado a
    sério pela equipe de professores, funcionários e, principalmente, pelo
    alunos. É direcionado para que os alunos se familiarizem com as
    Tecnologias de Informação e Comunicação (TIC) desde o jardim de
    infância. ``\emph{Robótica é uma coisa futurista. Daqui a 20 anos, a robótica
    vai estar integrada ao mundo. Aprendendo agora, não vou estar desfalcado
    de conhecimento lá na frente}'', acredita João Pedro, mostrando a
    fotografia do robô. Com certo ar de tristeza, sua colega da sétima
    série, Luara, explica que ``\emph{Godofredo estava condenado à morte cerebral
    desde o início}'', pois a turma seguinte precisava das peças para montar
    seu próprio invento. João Pedro destruiu o robô apagando o programa da
    memória do computador. No outro extremo dessa realidade à frente do
    mundo desenvolvido, estão escolas públicas que até hoje não têm
    computadores e jamais puderam organizar um laboratório de informática.
    Muitas empreenderam obras de infra-estrutura, à espera de equipamentos
    que nunca chegaram, como é o caso do Centro de Ensino Médio de Santa
    Maria, na periferia sul da capital federal. No Centro de Ensino
    Fundamental da Telebrasília, situado em uma favela, houve problemas com
    roubo de equipamentos, e a sala de informática terminou sendo
    desativada. Outras instituições não conseguiram sequer apresentar um
    computador às crianças.

    O Colégio Anima distribuiu pelas ruas de Brasília um outdoor anunciando
    que é a primeira escola do DF a oferecer robótica e educação tecnológica
    no currículo.

    Para o diretor, Bruno Castro Sousa, ``\emph{o mundo de hoje necessita de
    pessoas criativas, autônomas e flexíveis para lidar com as mudanças
    decorrentes da globalização e dos avanços tecnológicos}''.

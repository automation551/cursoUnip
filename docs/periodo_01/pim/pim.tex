%\documentclass[a4paper,12pt]{abnt}
\documentclass[a4paper,12pt]{report}
\usepackage[brazil]{babel}
\usepackage[utf8]{inputenc}

\begin{document}

    \author{Daniela Alvarenga                   \\RA A496JB-3
        \and João Paulo Dubas                   \\RA A54467-4
        \and Juliana Viana                      \\RA A437JH-6
        \and Leandro de Pinho Garcia            \\RA A59632-1
        \and Robson Miranda Alves de Oliveira   \\RA}
    \title{Projeto Integrado Multidisciplinar\\Primeiro Período}
    %\orientador{Prof. Msdo. Figueiredo}
    %\instituicao{Universidade Paulista (UNIP)\par
    %    Superior Tecnólogo em Análise e Desenvolvimento de Sistemas}
    %\local{Santos}
    \date{Santos\\2010}

    \maketitle

    \tableofcontents

    \chapter{Introdução}
    \label{ch:intro}

    \chapter[Educação versus Informática]{Relação entre Educação e Informática}
    \label{ch:edu_x_info}

    \chapter{Modelos de Sucesso}
    \label{ch:modelo_sucesso}

    \chapter[Laboratório]{Proposta do Laboratório de Informática}
    \label{ch:laboratorio}

        \section{Introdução}
        \label{sc:laboratorio_intro}
        Neste capítulo iremos abordar o estudo de caso para emprego de um
        laboratório de informática na escola de ensino médio
        \begin{bfseries}Colégio de Faz de Contas\end{bfseries}.

        Após introdução dos elementos característicos da escola e das
        necessidades do projeto, sera apresentada e discutida a solução
        proposta, custo envolvido e, como tal proposta será empregada no
        \begin{bfseries}Colégio Faz de Contas\end{bfseries}.

        \section{Estrutura \& Necessidades}
        \label{sc:laboratorio_estrutura}
        O Colégio Faz de Contas, situado à Rua do Sobe e Desce, sem
        número, na Terra do Nunca; conta com uma instalação de
        100m\begin{math}^{2}\end{math}, salas aclimatizadas e biblioteca.

            \subsection{Quantidade de alunos}
            Atualmente o Colégio Faz de Contas, atende a cerca de 500
            alunos, em todas as séries do Ensino Fundamental (1\textordfeminine
            à 8\textordfeminine séries), nos períodos matutino e vespertino.

            \subsection{Estrutura de ensino}
            Para permitir ao aluno um ensino integral\footnote{lembrar do
            colégio onde ocorreu o treinamento de Python} o colégio
            disponibiliza salas de estudo especiais e atividades extra
            curriculares.

            Além desta possibilidade, todos os professores prestam plantões
            após o horário de aula e, constantemente, os diretores do colégio
            promovem eventos para disseminação do conhecimento entre os
            alunos.

            \subsection{Bibliotecas}
            O Colégio Faz de Contas conta com uma sala de estudo reservada
            para uso pelos professores, em atividades de ensino, bem como com
            uma biblioteca para atender alunos e professores em suas pesquisas.

            \subsection{Salas de aula}
            O Colégio Faz de Contas dispões de salas de aula que atendem entre
            20 e 40 alunos (as turmas formadas são de no máximo 30 alunos).
            Estas salas dispõem de tv, dvd e cd player.

            Além das salas de aula, é disponível um espaço que tem por
            objetivo servir de espaço para o laboratório de informática.

            \subsection{Necessidades}

        \section{Propostas de ação}
        \label{sc:laboratorio_proposta}
        Compreendendo o colégio e quais suas necessidades, foram montadas
        propostas.

            \subsection{Proposta 1: \emph{Thin client}}
            Considerando a vida útil dos aparelhos eletrônicos e sua
            disponibilidade, uma forma interessante de tentar aproveitar de
            forma melhor as máquinas disponíveis e aumentar a disponibilidade
            do laboratório, envolve a combinação de servidores e \emph{thin
            client}.

            Além de tornarem os custos de manutenção inferiores àqueles
            obtidos em instalações compostas por computadores pessoais, esta
            solução, também proporciona redução nos custos de atualização do
            do software (uma vez que os custos de licença e suporte são
            menores com software livre) e atualização de hardware (já que as
            atualizações são relacionadas à estrutura de rede ou servidor e não
            há cada computador).

            Para atender às necessidades do Colégio Faz de Contas, serão
            necessários:

            \begin{itemize}

                \item 
            \end{itemize}

            \subsection{Proposta 2:}

            \subsection{Proposta 3:}

    \appendix
    \chapter{Requisitos do software}

        \section{Propósito}
        O software deverá seguir as seguintes orientações:

        \begin{enumerate}
            
            \item Permitir cadastrar uma turma

                \begin{itemize}
                    
                    \item Nome da turma

                    \item Ciclo da turma

                    \item Período da turma \{1: `matutino', 2: `vespertino',
                        3: `noturno'\}

                \end{itemize}

            \item Permitir cadastrar uma avaliação de uma turma

                \begin{itemize}
                    
                    \item Quantidade de alunos na turma

                \end{itemize}

            \item Permitir cadastrar diversas avaliações de alunos

                \begin{itemize}
                    
                    \item Quantidade de avaliações do aluno

                \end{itemize}
            
            \item A respeito do aluno serão cadastradas

                \begin{itemize}

                    \item Nome do aluno

                    \item Gênero do aluno \{1: `masculino', 2: `feminino'\}

                    \item Data de nascimento

                \end{itemize}

            \item A respeito da avaliação serão cadastradas

                \begin{itemize}

                    \item Data da avaliação

                    \item Massa corporal (kg)

                    \item Estatura (cm)

                    \item Circunferência da cintura (cm)

                    \item Circunferência do quadril (cm)

                \end{itemize}

        \end{enumerate}
        
        Ao professor será emitido um laudo com:

        \begin{itemize}

            \item Idade de cada aluno (em anos e em meses)

            \item Índice de massa corporal (IMC) de cada aluno

            \item Classificação em percentil do IMC pela idade

            \item Relação cintura/quadril (RCQ) de cada aluno

            \item Índice de conicidade (IC) de cada aluno

            \item Classificação da RCQ e IC de cada aluno

            \item Representação gráfica da massa corporal, estatura e IMC
            pela idade (caso tenhamos um protótipo)

        \end{itemize}

        No trabalho é interessante mostrar a ligação entre os assuntos
        envolvidos no desenvolvimento do aplicativo e as disciplinas que
        gostaríamos de envolver no processo de ensino (matemática, educação
        física e saúde).

        \section{Cálculos}

            \subsection{Índice de massa corporal (IMC)}
            O IMC ($kg/m^2$) é calculado de acordo com a equação:

            \begin{equation}
            \label{eqn:imc}
            IMC=\frac{MC}{H^2}
            \end{equation}

            Onde: $MC$ é a massa corporal em quilogramas e $H$ é a
            estatura em metros.

            \subsection{Relação cintura quadril (RCQ)}
            A RCQ é calculada de acordo com a equação:

            \begin{equation}
            \label{eqn:rcq}
            RCQ=\frac{CCT}{CQD}
            \end{equation}

            Onde: $CCT$ é a circunferência de cintura em centímetros e $CQD$
            é a circunferência de quadril em centímetros.
            
            \subsection{Índice de conicidade (IC)}
            O IC é calculado de acordo com a equação:
            
            \begin{equation}
            \label{eqn:ic}
            IC=\frac{CAB}{(0,109\times\sqrt{\frac{MC}{H}})}
            \end{equation}

\end{document}

\documentclass[a4paper,10pt]{report}
\usepackage[brazil]{babel}
\usepackage[utf8]{inputenc}

\begin{document}

    \title{Projeto Integrado Multidisciplinar\\Primeiro Período}
    \author{Daniela Alvarenga   \\RA
        \and João Paulo Dubas   \\RA
        \and Juliana Viana      \\RA
        \and Robson \ldots      \\RA}
    \date{Santos\\2010}

    \maketitle

    \tableofcontents

    \chapter{Introdução}

    \chapter[Educação versus Informática]{Relação entre Educação e Informática}

    \chapter{Modelos de Sucesso}

    \chapter[Laboratório]{Proposta do Laboratório de Informática}

        \section{Introdução}

            Neste capítulo iremos abordar o estudo de caso para emprego de um
            laboratório de informática na escola de ensino médio
            \begin{bfseries}Colégio de Faz de Contas\end{bfseries}.

            Após introdução dos elementos característicos da escola e das
            necessidades do projeto, sera apresentada e discutida a solução
            proposta, custo envolvido e, como tal proposta será empregada no
            \begin{bfseries}Colégio Faz de Contas\end{bfseries}.

        \section{Estrutura \& Necessidades}

            \subsection{Quantidade de alunos}

            \subsection{Estrutura de ensino}

            \subsection{Bibliotecas}

            \subsection{Salas de aula}

        \section{Propostas de ação}

            \subsection{Proposta 1:}

            \subsection{Proposta 2:}

            \subsection{Proposta 3:}

\end{document}

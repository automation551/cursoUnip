%\documentclass[a4paper,12pt]{abnt}
\documentclass[a4paper,12pt]{report}
\usepackage[brazil]{babel}
\usepackage[utf8]{inputenc}

\begin{document}

    \author{Daniela Alvarenga                   \\RA A496JB-3
        \and João Paulo Dubas                   \\RA A54467-4
        \and Juliana Viana                      \\RA A437JH-6
        \and Leandro de Pinho Garcia            \\RA A59632-1
        \and Robson Miranda Alves de Oliveira   \\RA}
    \title{Projeto Integrado Multidisciplinar\\Primeiro Período}
    %\orientador{Prof. Msdo. Figueiredo}
    %\instituicao{Universidade Paulista (UNIP)\par
    %    Superior Tecnólogo em Análise e Desenvolvimento de Sistemas}
    %\local{Santos}
    \date{Santos\\2010}

    \maketitle

    \tableofcontents

    \chapter{Introdução}
    \label{ch:intro}

    \chapter[Educação versus Informática]{Relação entre Educação e Informática}
    \label{ch:edu_x_info}
    (2) A Informática vem adquirindo cada vez mais relevância no cenário
    educacional e, enquanto disciplina, (1) a Informática tem como objetivos
    introduzir os alunos aos principais sistemas operacionais e aplicativos
    computacionais usados no meio científico-acadêmico, bem como fornecer
    conceitos básicos no desenvolvimento de programas.

    (2) Porém, vivemos em um mundo tecnológico, onde a Informática é uma das
    peças principais e, conceber a Informática apenas como disciplina, é ignorar
    sua atuação em nossas vidas. A Informática deve ser levada para toda escola
    ao invés de ser restringida a uma sala, presa em um horário fixo e sob a
    responsabilidade de um único professor, pois isso limita o processo de
    desenvolvimento da escola como um todo e se perde assim a oportunidade de
    fortalecer o processo pedagógico.

    A Informática então, também pode e deve ser utilizada como instrumento de
    aprendizagem e, nesse sentido, a educação vem passando por mudanças
    estruturais e funcionais frente a essa nova tecnologia.

    Houve época em que era necessário justificar a introdução da Informática na
    escola. Hoje, já existe consenso quanto à sua importância e, o principal
    objetivo defendido hoje ao adaptar a Informática ao currículo escolar, está
    na utilização do computador como instrumento de apoio às matérias e aos
    conteúdos lecionados, além da função de preparar os alunos para uma
    sociedade informatizada.

    No começo, quando as escolas começaram a introduzir a Informática no ensino,
    percebeu-se, pela pouca experiência com essa tecnologia, um processo um
    pouco caótico. Muitas escolas introduziram em seu currículo o ensino da
    Informática com o pretexto da modernidade. Mas não identificavam exatamente
    o que fazer nessa aula e até mesmo quem poderia dar essas aulas. A
    princípio, contrataram técnicos que tinham como missão ensinar a Informática
    Básica apenas, com quase nenhum vínculo com as disciplinas, cujos objetivos
    principais eram o contato com a nova tecnologia e oferecer a formação
    tecnológica necessária para o futuro profissional na sociedade.

    Porém, com o passar do tempo, algumas escolas, percebendo o potencial dessa
    ferramenta, introduziram a Informática Educativa, que, além de promover o
    contato com o computador, tem como objetivo a utilização dessa ferramenta
    como instrumento de apoio às matérias e aos conteúdos lecionados. Esse apoio
    é vinculado a uma disciplina de Informática, que tem a função de oferecer os
    recursos necessários para que os alunos apresentem o conteúdo das outras
    disciplinas.

    (3) Para compreendermos melhor as diferentes formas como o computador tem
    chegado à escola e qual delas melhor caracteriza a Informática Educativa,
    BORGES NETO (1999) faz uma classificação sobre a iniciação do computador na
    escola e seus diferentes usos: Informática Aplicada à Educação, Informática
    na Educação, Informática Educacional e Informática Educativa. Vejamos a
    seguir em que aspectos elas se diferenciam:

    \begin{itemize}

        \item A Informática Aplicada à Educação caracteriza-se pelo uso do
        computador em trabalhos burocráticos da escola, como, por exemplo,
        controle de matrículas, de notas, folhas de pagamento, tabelas,
        digitação de ofícios, relatórios e outros documentos internos da escola.

        \item A Informática na Educação corresponde ao uso do computador através
        de softwares de apoio e suporte à educação como tutoriais, livros
        multimídias, buscas na Internet e o uso de outros aplicativos em geral.
        Nesse caso, geralmente o aluno vai ao laboratório para aulas de reforço
        ou para praticar atividades de Informática Básica que, na maioria das
        vezes, não apresentam nenhum vínculo com os conhecimentos trabalhados em
        sala de aula.

        \item A informática Educacional indica o uso do computador como
        ferramenta auxiliar na resolução de problemas. Nesse caso, as atividades
        desenvolvidas no laboratório são resultantes ou interligadas a projetos.
        Os alunos podem fazer uso dos recursos informáticos disponíveis. Aqui,
        eles executam as atividades, trabalhando sozinhos no computador ou com
        auxilio de um professor ou monitor de informática. Assim, por mais bem
        planejadas que sejam as atividades geradas pelos projetos, a
        aprendizagem dos conteúdos acaba não se processando de maneira ideal,
        pois não há intervenções do professor especialista (Português,
        Matemática, etc.) para conduzir a aprendizagem.

        \item A Informática Educativa se caracteriza pelo uso pleno da
        Informática como um instrumento a mais para o professor utilizar em suas
        aulas. Aqui, o professor especialista deve utilizar os recursos
        informáticos disponíveis, explorando as potencialidades oferecidas pelo
        computador e pelos softwares, aproveitando o máximo possível suas
        capacidades para simular, praticar ou evidenciar situações, geralmente
        de impossível apreensão desta maneira por outras mídias. Nesse modelo, a
        Informática exerce o papel de agente colaborador e meio didático na
        propagação do conhecimento, posta à disposição da educação, através do
        qual o professor interage com seus alunos na construção do conhecimento
        objetivado.

    \end{itemize}

    De acordo com a classificação acima, podemos perceber que a Informática
    poderá ser utilizada na escola através de concepções diferentes em vários
    tipos de atividades. Entendemos que as quatro concepções são importantes e
    não se contradizem, mais se complementam e, que, a Informática Educativa
    deve ser considerada como prioritária entre as quatro modalidades, pois, só
    através dela, poderão ocorrer as contribuições mais significativas para a
    construção do conhecimento por parte dos alunos.

    (2) A Informática deve habilitar e dar oportunidade ao aluno de adquirir
    novos conhecimentos, facilitar o processo ensino/aprendizagem, ser um
    complemento de conteúdos curriculares visando o desenvolvimento integral do
    indivíduo.

    O acesso à Informática deve ser visto como um direito e, portanto, nas
    escolas públicas e particulares o estudante deve poder usufruir de uma
    educação que no momento atual inclua, no mínimo, uma alfabetização
    tecnológica. Tal alfabetização deve ser vista não como um curso de
    Informática, mas sim como um aprender a ler essa nova mídia. Assim, o
    computador deve estar inserido em atividades essenciais, tais como aprender
    a ler, escrever, compreender textos, entender gráficos, contar, desenvolver
    noções espaciais, etc. E, nesse sentido, a Informática na escola passa a ser
    parte da resposta a questões ligadas à cidadania.

    \chapter{Modelos de Sucesso}
    \label{ch:modelo_sucesso}

    \chapter[Laboratório]{Proposta do Laboratório de Informática}
    \label{ch:laboratorio}

        \section{Introdução}
        \label{sc:laboratorio_intro}
        Neste capítulo iremos abordar o estudo de caso para emprego de um
        laboratório de informática na escola de ensino médio
        \begin{bfseries}Colégio de Faz de Contas\end{bfseries}.

        Após introdução dos elementos característicos da escola e das
        necessidades do projeto, sera apresentada e discutida a solução
        proposta, custo envolvido e, como tal proposta será empregada no
        \begin{bfseries}Colégio Faz de Contas\end{bfseries}.

        \section{Estrutura \& Necessidades}
        \label{sc:laboratorio_estrutura}
        O Colégio Faz de Contas, situado à Rua do Sobe e Desce, sem
        número, na Terra do Nunca; conta com uma instalação de
        100m\begin{math}^{2}\end{math}, salas aclimatizadas e biblioteca.

            \subsection{Quantidade de alunos}
            Atualmente o Colégio Faz de Contas, atende a cerca de 500
            alunos, em todas as séries do Ensino Fundamental (1\textordfeminine
            à 8\textordfeminine séries), nos períodos matutino e vespertino.

            \subsection{Estrutura de ensino}
            Para permitir ao aluno um ensino integral\footnote{lembrar do
            colégio onde ocorreu o treinamento de Python} o colégio
            disponibiliza salas de estudo especiais e atividades extra
            curriculares.

            Além desta possibilidade, todos os professores prestam plantões
            após o horário de aula e, constantemente, os diretores do colégio
            promovem eventos para disseminação do conhecimento entre os
            alunos.

            \subsection{Bibliotecas}
            O Colégio Faz de Contas conta com uma sala de estudo reservada
            para uso pelos professores, em atividades de ensino, bem como com
            uma biblioteca para atender alunos e professores em suas pesquisas.

            \subsection{Salas de aula}
            O Colégio Faz de Contas dispões de salas de aula que atendem entre
            20 e 40 alunos (as turmas formadas são de no máximo 30 alunos).
            Estas salas dispõem de tv, dvd e cd player.

            Além das salas de aula, é disponível um espaço que tem por
            objetivo servir de espaço para o laboratório de informática.

            \subsection{Necessidades}

        \section{Propostas de ação}
        \label{sc:laboratorio_proposta}
        Compreendendo o colégio e quais suas necessidades, foram montadas
        as propostas descritas abaixo:

            \subsection{Proposta 1: \emph{Thin client}}
            Considerando a vida útil dos aparelhos eletrônicos e sua
            disponibilidade, uma forma interessante de tentar aproveitar de
            forma melhor as máquinas disponíveis e aumentar a disponibilidade
            do laboratório, envolve a combinação de servidores e \emph{thin
            client}.

            Além de tornarem os custos de manutenção inferiores àqueles
            obtidos em instalações compostas por computadores pessoais, esta
            solução, também proporciona redução nos custos de atualização do
            do software (uma vez que os custos de licença e suporte são
            menores com software livre) e atualização de hardware (já que as
            atualizações são relacionadas à estrutura de rede ou servidor e não
            há cada computador).

            Para atender às necessidades do Colégio Faz de Contas, serão
            necessários:

            \begin{description}

                \item[Servidor] Serão utilizados quatro servidores para, sendo
                dois para o departamento de biblioteca e dois para o
                laboratório de informática. Dessa forma, teremos um servidor
                de reserva, caso seja necessário realizar manutenção ou
                haja falha de uma das máquinas.

                \item[\emph{Storage}] Será empregado um blade da HP para
                armazenamento dos dados relativos aos usuários, tais como,
                perfis, arquivos salvos, entre outros.

                \item[\emph{Thin client}]

            \end{description}

            \subsection{Proposta 2:}

            \subsection{Proposta 3:}

    \appendix

    \chapter{Reuniões de grupo}

    \begin{description}

        \item[02 de Março] Definição do grupo e composição do tema.

        \item[09 de Março] Definição dos tópicos a serem discutidos, proposta
        inicial de trabalho e montagem do cronograma.

        \item[16 de Março] Adequação dos tópicos a serem discutidos, organização
        do cronograma.

        \item[23 de Março] Definição dos requisitos para programa a ser
        apresentado.

        \item[30 de Março] Inicio do trabalho de revisão de bibliografia.

        \item[06 de Abril] Inicio da montagem para proposta do laboratório de
        informática.

        \item[13 de Abril] Levantamento de necessidades do colégio \emph{Faz de
        contas}.

        \item[20 de Abril] Inicio da organização das referências revisadas e
        escrita do texto provisório.

        \item[27 de Abril] Revisão dos conteúdos já montados e progresso do
        trabalho.

        \item[04 de Maio] Formatação do texto relativo ao capítulo 2.

        \item[11 de Maio] Divisão do trabalho para o novo integrante do grupo e
        revisão do progresso.

    \end{description}

    \chapter{Requisitos do software}

        \section{Propósito}
        O software deverá seguir as seguintes orientações:

        \begin{enumerate}

            \item Permitir cadastrar uma turma

                \begin{itemize}

                    \item Nome da turma

                    \item Ciclo da turma

                    \item Período da turma \{1: `matutino', 2: `vespertino',
                        3: `noturno'\}

                \end{itemize}

            \item Permitir cadastrar uma avaliação de uma turma

                \begin{itemize}

                    \item Quantidade de alunos na turma

                \end{itemize}

            \item Permitir cadastrar diversas avaliações de alunos

                \begin{itemize}

                    \item Quantidade de avaliações do aluno

                \end{itemize}

            \item A respeito do aluno serão cadastradas

                \begin{itemize}

                    \item Nome do aluno

                    \item Gênero do aluno \{1: `masculino', 2: `feminino'\}

                    \item Data de nascimento

                \end{itemize}

            \item A respeito da avaliação serão cadastradas

                \begin{itemize}

                    \item Data da avaliação

                    \item Massa corporal (kg)

                    \item Estatura (cm)

                    \item Circunferência da cintura (cm)

                    \item Circunferência do quadril (cm)

                \end{itemize}

        \end{enumerate}

        Ao professor será emitido um laudo com:

        \begin{itemize}

            \item Idade de cada aluno (em anos e em meses)

            \item Índice de massa corporal (IMC) de cada aluno

            \item Classificação em percentil do IMC pela idade

            \item Relação cintura/quadril (RCQ) de cada aluno

            \item Índice de conicidade (IC) de cada aluno

            \item Classificação da RCQ e IC de cada aluno

            \item Representação gráfica da massa corporal, estatura e IMC
            pela idade (caso tenhamos um protótipo)

        \end{itemize}

        No trabalho é interessante mostrar a ligação entre os assuntos
        envolvidos no desenvolvimento do aplicativo e as disciplinas que
        gostaríamos de envolver no processo de ensino (matemática, educação
        física e saúde).

        \section{Cálculos}

            \subsection{Índice de massa corporal (IMC)}
            O IMC ($kg/m^2$) é calculado de acordo com a equação:

            \begin{equation}
            \label{eqn:imc}
            IMC=\frac{MC}{H^2}
            \end{equation}

            Onde: $MC$ é a massa corporal em quilogramas e $H$ é a
            estatura em metros.

            \subsection{Relação cintura quadril (RCQ)}
            A RCQ é calculada de acordo com a equação:

            \begin{equation}
            \label{eqn:rcq}
            RCQ=\frac{CCT}{CQD}
            \end{equation}

            Onde: $CCT$ é a circunferência de cintura em centímetros e $CQD$
            é a circunferência de quadril em centímetros.

            \subsection{Índice de conicidade (IC)}
            O IC é calculado de acordo com a equação:

            \begin{equation}
            \label{eqn:ic}
            IC=\frac{CAB}{(0,109\times\sqrt{\frac{MC}{H}})}
            \end{equation}

\end{document}

\chapter[Laboratório]{Proposta do Laboratório de Informática}
\label{ch:laboratorio}

    \section{Introdução}
    \label{sc:laboratorio_intro}
    Neste capítulo iremos abordar o estudo de caso para emprego de um
    laboratório de informática na escola de ensino médio
    \begin{bfseries}Colégio de Faz de Contas\end{bfseries}.

    Após introdução dos elementos característicos da escola e das
    necessidades do projeto, sera apresentada e discutida a solução
    proposta, custo envolvido e, como tal proposta será empregada no
    \begin{bfseries}Colégio Faz de Contas\end{bfseries}.

    \section{Estrutura \& Necessidades}
    \label{sc:laboratorio_estrutura}
    O Colégio Faz de Contas, situado à Rua do Sobe e Desce, sem
    número, na Terra do Nunca; conta com uma instalação de
    100m\begin{math}^{2}\end{math}, salas aclimatizadas e biblioteca.

        \subsection{Quantidade de alunos}
        Atualmente o Colégio Faz de Contas, atende a cerca de 500
        alunos, em todas as séries do Ensino Fundamental (1\textordfeminine
        à 8\textordfeminine séries), nos períodos matutino e vespertino.

        \subsection{Estrutura de ensino}
        Para permitir ao aluno um ensino integral o colégio
        disponibiliza salas de estudo especiais e atividades extra
        curriculares.

        Além desta possibilidade, todos os professores prestam plantões
        após o horário de aula e, constantemente, os diretores do colégio
        promovem eventos para disseminação do conhecimento entre os
        alunos.

        \subsection{Bibliotecas}
        O Colégio Faz de Contas conta com uma sala de estudo reservada
        para uso pelos professores, em atividades de ensino, bem como com
        uma biblioteca para atender alunos e professores em suas pesquisas.

        \subsection{Salas de aula}
        O Colégio Faz de Contas dispões de salas de aula que atendem entre
        20 e 40 alunos (as turmas formadas são de no máximo 30 alunos).
        Estas salas dispõem de tv, dvd e cd player.

        Além das salas de aula, é disponível um espaço que tem por
        objetivo servir de espaço para o laboratório de informática.

        \subsection{Necessidades}
        Apesar de disponiblizar recursos de áudio e vídeo para seus alunos e
        professores, o que enriquece o aprendizando dos alunos e as
        possibilidades educacionais para os professores, a diretorio e o
        coordenador pedagógico do Colégio Faz de Contas, percebem atualmente
        que seu pontencial pedagógico e de formação do cidadão ficam
        prejudicados pela ausência de um laboratório de informática.

        Para tentar sanar este problema, foi pensado em um espaço que
        deveria prover ao corpo discente do colégio acesso à informática:

        \begin{description}

            \item[Laboratório] Deverá atender a 50 alunos, tendo como meta
            permitir mais uma ferramenta educacional para o professor. Nesse
            laboratório, professor e aluno terão acesso à internet, porém
            restrições deverão ser aplicadas, tais como:

            \begin{enumerate}

                \item Acesso a sites de redes socias.

                \item Acesso a sites de novelas e/ou \emph{fofocas}.

            \end{enumerate}

        \end{description}

    \section{Propostas de ação}
    \label{sc:laboratorio_proposta}
    Compreendendo o colégio e quais suas necessidades, foram montadas
    as propostas descritas abaixo:

        \subsection{Proposta: \emph{Thin client}}
        Considerando a vida útil dos aparelhos eletrônicos e sua
        disponibilidade, uma forma interessante de tentar aproveitar de
        forma melhor as máquinas disponíveis e aumentar a disponibilidade
        do laboratório, envolve a combinação de servidores e \emph{thin
        client}.

        Além de tornarem os custos de manutenção inferiores àqueles
        obtidos em instalações compostas por computadores pessoais, esta
        solução, também proporciona redução nos custos de atualização do
        do software (uma vez que os custos de licença e suporte são
        menores com software livre) e atualização de hardware (já que as
        atualizações são relacionadas à estrutura de rede ou servidor e não
        há cada computador, ou seja, a administração dos computadores é
        centralizada no servidor) \cite{richards:2007}.

        Este sistema (servidor - \emph{thin client})funciona da seguinte
        forma \cite{techlearning:2008}:

        \begin{itemize}

            \item A aplicação é executada completamente no servidor.

            \item No lado do servidor, a imagem é capturada, comprimida,
            criptografada e enviada para o \emph{thin client}, onde será
            exibida.

            \item No lado do \emph{thin client}, o teclado e os movimentos
            do mouse são capturados e transmitidos para o servidor.

            \item Todas as transações são realizadas através Protocolos
            de Desktop Remoto (\emph{Remote Desktop Protocol}, RDP) ou
            protocolo Arquitertura Computacional Independente
            (\emph{Independet Computing Architecture}, ICA), ambos de baixa
            largura de banda.

        \end{itemize}

        Para atender às necessidades do Colégio Faz de Contas, serão
        necessários:

        \begin{description}

            \item[Servidor] Serão utilizados dois servidores, em cluster
            com balanceamento de carga. Permitindo assim melhor divisão do
            trabalho e maior velocidade no momento servir às requisições
            feitas pelos \emph{thin client}.

            Os servidores, serão \emph{HP ProLiant BL280c G6}, com as
            seguintes especificações:

            \begin{description}

                \item[Processador] Dois Intel Xeon Quad-Core E5504.

                \item[Memória] 8 pentes PC3-10600E de 2GB cada.

                \item[Disco] 2 discos rigidos de 160GB cada, modelo
                \emph{3G Non-Hot Plug SATA}, 7200rpm.

                \item[Sistema Operacional] Debian 5.0 GNU/Linux, com sistema
                de virtualização por meio do software Xen, onde serão
                servidos o sistema Edubuntu 10.04, que apresenta soluções
                integradas para educação.

            \end{description}

            O custo destes servidores será de US\$ 3169,00 cada.

            \item[\emph{Storage}] Serão empregados dois sistemas de discos
            para \emph{backup} e armazenamento de perfis e configurações
            dos usuários.

            Para tal serão adquiridos dois sistemas de backup HP RDX500
            com disco interno removível de 500GB, modelo AJ934SB. Com custo
            de US\$ 649,00 cada.

            \item[\emph{Thin client}] Serão utilizados 51 terminais
            para o laboratório de
            informática. Ester terminais serão \emph{stateless} e
            \emph{fanless}, ou seja, sem partes móveis, o que garante maior
            durabilidade e reduzida necessidade de suporte \cite{richards:2007}.

            Serão utilizadas a unidade \emph{thin client} da HP Compaq,
            modelo t5325.

        \end{description}


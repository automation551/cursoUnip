\chapter*{Introdu��o}
\label{ch:intro}
Atualmente a inform�tica esta ligada a todos os ramos de atua��o do homem e
incondicionalmente ligada ao cotidiano das pessoas.

Desprezar ou n�o levar em conta seu potencial educacional � prejudicial �
forma��o dos alunos atualmente.

Dessa forma, neste trabalho s�o propostos a cria��o de um laborat�rio de
inform�tica de baixo custo, que permita ao corpo doscente e discente fazer
melhor uso dessa ferramenta no processo de ensino-aprendizagem. Al�m disso
ser� apresentado um software para auxiliar as disciplinas de matem�tica e
sa�de (tema transversal) por meio do c�lculo do �ndice de massa corporal.

Para atingir estes objetivos, ser�o discutidos a seguir alguns aspectos da
rela��o entre inform�tica e educa��o e alguns casos de sucesso no emprego
da inform�tica nas escolas.
